\documentclass[UTF8]{ctexart}
\usepackage{float}
\usepackage{listings}
\usepackage{graphicx}
\usepackage{xeCJK}
\usepackage{xcolor}
\usepackage{ulem}
\usepackage{CJKfntef}
\usepackage{setspace}
\usepackage{amsmath}
\lstset{
   basicstyle          =   \ttfamily,          % 基本代码风格
   keywordstyle        =   \ttfamily\bfseries,          % 关键字风格
   commentstyle        =   \rmfamily\itshape,  % 注释的风格,斜体
   stringstyle         =   \ttfamily,  % 字符串风格
   flexiblecolumns,                % 别问为什么,加上这个
   numbers             =   left,   % 行号的位置在左边
   showspaces          =   false,  % 是否显示空格,显示了有点乱,所以不现实了
   numberstyle         =   \zihao{-5}	tfamily,    % 行号的样式,小五号,tt等宽字体
   showstringspaces    =   false,
   captionpos          =   t,      % 这段代码的名字所呈现的位置,t指的是top上面
   frame               =   lrtb,   % 显示边框
   breaklines      =   true,   % 自动换行,建议不要写太长的行
   columns         =   fixed,  % 如果不加这一句,字间距就不固定,很丑,必须加
}

\title{离散数学第二次作业}
\author{贠启豪 19375168}
\date{\today}

\begin{document}
   \maketitle
    \begin{enumerate}
        \item 用真值表证明下列公式
        \begin{enumerate}
            \item $p\wedge (q\oplus r)\Leftrightarrow (p \wedge q)\oplus (p\wedge r) $
            
            解:
            \begin{table}[H]
                \centering
                \begin{tabular}{|cccccccc|}
                    \hline
                    $p$ & $q$ & $r$ & $q\oplus r$ & $p\wedge (q\oplus r)$ & $p\wedge q$ & $p\wedge r$ & $(p \wedge q)\oplus (p\wedge r)$ \\
                    \hline
                    0 & 0 & 0 & 0 & 0 & 0 & 0 & 0 \\ 
                    \hline
                    0 & 0 & 1 & 1 & 0 & 0 & 0 & 0 \\
                    \hline
                    0 & 1 & 0 & 1 & 0 & 0 & 0 & 0 \\
                    \hline
                    0 & 1 & 1 & 0 & 0 & 0 & 0 & 0 \\
                    \hline
                    1 & 0 & 0 & 0 & 0 & 0 & 0 & 0 \\
                    \hline
                    1 & 0 & 1 & 1 & 1 & 0 & 1 & 1 \\
                    \hline
                    1 & 1 & 0 & 1 & 1 & 1 & 0 & 1 \\
                    \hline
                    1 & 1 & 1 & 0 & 0 & 1 & 1 & 0 \\
                    \hline
                \end{tabular}
            \end{table}
            真值表如上图所示,可以看出,$p\wedge (q\oplus r)$
            的取值总和$(p \wedge q)\oplus (p\wedge r)$
            相同,故有
            \[
                p\wedge (q\oplus r)\Leftrightarrow (p \wedge q)\oplus (p\wedge r)
            \]

            \item $p\oplus 1 \Leftrightarrow \neg p$
            
            解:
            \begin{table}[H]
                \centering
                \begin{tabular}{|ccc|}
                    \hline
                    $p$ & $p\oplus 1$ & $\neg p$ \\
                    \hline
                    0 & 1 & 1 \\ 
                    \hline
                    1 & 0 & 0 \\
                    \hline
                \end{tabular}
            \end{table}
            真值表如上图所示,可以看出,$p\oplus 1$
            的取值总和$\neg p$
            相同,故有
            \[
                p\oplus 1 \Leftrightarrow \neg p
            \]

            \item $p \vee (p\wedge q)\Leftrightarrow p$
            
            解:

            \begin{table}[H]
                \centering
                \begin{tabular}{|ccccc|}
                    \hline
                    $p$ & $q$ & $p\wedge q$ & $p\vee (p\wedge q)$ & $p$ \\
                    \hline
                    0 & 0 & 0 &0 & 0 \\ 
                    \hline
                    0 & 1 & 0 &0 & 0 \\
                    \hline
                    1 & 0 & 0 &1 & 1 \\
                    \hline
                    1 & 1 & 1 &1 & 1 \\
                    \hline
                \end{tabular}
            \end{table}
            真值表如上图所示,可以看出,$p\vee (p\wedge q)$
            的取值总和$p$
            相同,故有
            \[
                p \vee (p\wedge q)\Leftrightarrow p
            \]

            \item $p\oplus q \Leftrightarrow \neg(p\leftrightarrow q)$
            
            解:

            \begin{table}[H]
                \centering
                \begin{tabular}{|ccccc|}
                    \hline
                    $p$ & $q$ & $p\oplus q$ & $p \leftrightarrow q$ & $\neg (p\leftrightarrow q)$ \\
                    \hline
                    0 & 0 & 0 &1 & 0 \\ 
                    \hline
                    0 & 1 & 1 &0 & 1 \\
                    \hline
                    1 & 0 & 1 &0 & 1 \\
                    \hline
                    1 & 1 & 0 &1 & 0 \\
                    \hline
                \end{tabular}
            \end{table}
            真值表如上图所示,可以看出,$p\oplus q$
            的取值总和$\neg (p \leftrightarrow q)$
            相同,故有
            \[
                p\oplus q \Leftrightarrow \neg(p\leftrightarrow q)
            \]
        \end{enumerate}



    \item 用等值演算证明以下等值式
        \begin{enumerate}
            \item $p \rightarrow (q \rightarrow r) \Leftrightarrow q \rightarrow ( p \rightarrow r)$
            
            \[  
                \begin{aligned}
                &\mathrel{\phantom{=}}p\rightarrow q(p\rightarrow r)\\
                 &\Leftrightarrow \neg p \vee (\neg q \vee r)\\
                 &\Leftrightarrow (\neg p \vee \neg q)\vee r\\
                 &\Leftrightarrow \neg q \vee (\neg p \vee r)\\
                 &\Leftrightarrow q \rightarrow (p \rightarrow r)\\
                \end{aligned}
             \]

            \item $( p \rightarrow q) \wedge ( p \rightarrow r) \Leftrightarrow p \rightarrow q \wedge r$
            \[
                \begin{aligned}
                    &\mathrel{\phantom{=}}(p\rightarrow q)\wedge (q\rightarrow r)\\
                    &\Leftrightarrow (\neg p \vee q)\wedge (\neg p \vee r)\\
                    &\Leftrightarrow \neg p \vee (q \wedge r)\\
                    &\Leftrightarrow p\rightarrow q \wedge r
                \end{aligned}
            \]
            
            
            \item  $( p \rightarrow q) \vee (r \rightarrow q) \Leftrightarrow p \wedge r \rightarrow q$
            
            \[
                \begin{aligned}
                    &\mathrel{\phantom{=}} (p\rightarrow q)\wedge (r \rightarrow q)\\
                    &\Leftrightarrow (\neg p \vee q)\vee (\neg r \vee q)\\
                    &\Leftrightarrow \neg p \vee \neg r \vee q\\
                    &\Leftrightarrow \neg(p\wedge r)\vee q\\
                    &\Leftrightarrow p\wedge r \rightarrow q\\
                \end{aligned}
            \]
            
            \item  $p \rightarrow (q \rightarrow p) \Leftrightarrow \neg p \rightarrow ( p \rightarrow q)$
            
            \[
                \begin{aligned}
                    &\mathrel{\phantom{=}} p\rightarrow (q\rightarrow p)\\
                    &\Leftrightarrow \neg p \vee (\neg q \vee p)\\
                    &\Leftrightarrow 1\\
                    &\Leftrightarrow p \vee \neg p \vee q \\
                    &\Leftrightarrow \neg \neg p \vee (\neg p \vee q)\\
                    &\Leftrightarrow \neg p \rightarrow (p \rightarrow q)\\
                \end{aligned}
            \]
            
            \item  $( p \rightarrow q) \wedge (r \rightarrow q) \Leftrightarrow p \vee r \rightarrow q$
            \[
                \begin{aligned}
                    &\mathrel{\phantom{=}} (p\rightarrow q) \wedge (r\wedge q)\\
                    &\Leftrightarrow (\neg p \vee q) \wedge (\neg r \vee q)\\
                    &\Leftrightarrow (\neg p \wedge \neg r) \vee q\\
                    &\Leftrightarrow \neg(p \vee r)\vee q\\
                    &\Leftrightarrow (p\vee r)\rightarrow q\\
                \end{aligned}
            \]
            \item  $\neg( p \leftrightarrow q) \Leftrightarrow p \leftrightarrow \neg q$
            \[
                \begin{aligned}
                    &\mathrel{\phantom{=}}\neg (p\leftrightarrow q)\\
                    &\Leftrightarrow p \oplus q\\
                    &\Leftrightarrow (p \oplus q )\oplus (1 \oplus 1)\\
                    &\Leftrightarrow (p \oplus (q\oplus 1))\oplus 1\\
                    &\Leftrightarrow \neg (p \oplus \neg q)\\
                    &\Leftrightarrow \neg \neg (p \leftrightarrow \neg q)\\
                    &\Leftrightarrow p \leftrightarrow \neg q\\
                \end{aligned}    
            \]
        \end{enumerate}
    
    \item 用等值演算证明以下公式是永真式
    \begin{enumerate}
        \item $( q \rightarrow  p ) \wedge  ( \neg p \rightarrow  q ) \leftrightarrow p$
        \[    
            \begin{aligned}
                &\mathrel{\phantom{=}} (q\rightarrow p)\wedge (\neg p \rightarrow q) \leftrightarrow p\\
                &\Leftrightarrow (\neg q \vee p)\wedge (p \vee q) \leftrightarrow p\\
                &\Leftrightarrow (\neg q \wedge q)\vee p \leftrightarrow p\\
                &\Leftrightarrow 0\wedge p \leftrightarrow p\\
                &\Leftrightarrow p\wedge p\\
                &\Leftrightarrow 1\\
            \end{aligned}
        \]

        \item $( p \rightarrow  q) \wedge  (r \rightarrow  s) \rightarrow  ( p \wedge  r \rightarrow  q \wedge  s)$
        \[
            \begin{aligned}
                &\mathrel{\phantom{=}}(p\rightarrow q)\wedge (r \rightarrow s)\rightarrow(p\wedge r \rightarrow q\wedge s)\\
                &\Leftrightarrow \neg((\neg p \vee q)\wedge (\neg r\vee s))\vee(\neg(p\wedge r)\vee (q \wedge s))\\
                &\Leftrightarrow \neg(\neg p\vee q) \vee \neg (\neg r \vee s)\vee (\neg p \vee \neg r \vee (q \wedge s))\\
                &\Leftrightarrow (p\wedge \neg q) \vee (r \wedge \neg s) \vee \neg p \vee \neg r \vee (q \wedge s)\\
                &\Leftrightarrow \neg p \vee \neg q \vee \neg r \vee \neg s \vee (q\wedge s)\\
                &\Leftrightarrow \neg(q\wedge s) \vee (q\wedge s) \vee \neg(p \wedge r)\\
                &\Leftrightarrow 1
            \end{aligned}
        \]
         
        
        \item $( p \rightarrow  q) \vee  ( p \rightarrow  r) \vee  ( p \rightarrow  s) \rightarrow  ( p \rightarrow  q \vee  r \vee  s )$
        \[
            \begin{aligned}
                &\mathrel{\phantom{=}}(p\rightarrow q)\vee (p \rightarrow r)\vee (p \rightarrow s)\rightarrow (p\rightarrow q\vee r \vee s)\\
                &\Leftrightarrow (\neg p \vee q \vee \neg p \vee r \vee \neg p \vee s )\rightarrow (\neg p \vee q \vee r \vee s)\\
                &\Leftrightarrow (\neg p \vee q \vee r \vee s )\rightarrow (\neg p \vee q \vee r \vee s)\\
                &\Leftrightarrow 1\\
            \end{aligned}
        \]
         
        
        \item  $( p \vee  q \rightarrow  r) \rightarrow  ( p \rightarrow  r) \vee  (q \rightarrow  r)$
        \[
            \begin{aligned}
                &\mathrel{\phantom{=}} (p\vee q \rightarrow r)\rightarrow (p\rightarrow r)\vee(q\rightarrow r)\\
                &\Leftrightarrow \neg(\neg (p\vee q)\vee r)\vee(\neg p \vee r \vee \neg q \vee r)\\
                &\Leftrightarrow ((p\vee q)\wedge \neg r) \vee (\neg p \vee \neg q \vee r)\\
                &\Leftrightarrow (p\vee q \vee \neg p \vee \neg q \vee r)\wedge(\neg r \vee r \vee \neg p \vee \neg q)\\
                &\Leftrightarrow 1\wedge 1\\
                &\Leftrightarrow 1\\
            \end{aligned}
        \]
    \end{enumerate}
    \item 用等值演算证明以下公式是永假式
    \begin{enumerate}
        \item $(q \rightarrow p) \wedge  (\neg p \rightarrow q) \leftrightarrow \neg p$
        \[
            \begin{aligned}
                &\mathrel{\phantom{=}}(q \rightarrow p) \wedge  (\neg p \rightarrow q) \leftrightarrow \neg p\\
                &\Leftrightarrow (\neg q \vee p)\wedge (p \vee q)\leftrightarrow \neg p\\
                &\Leftrightarrow (\neg q \vee q )\wedge p \leftrightarrow \neg p\\
                &\Leftrightarrow 1\wedge p \leftrightarrow \neg p\\
                &\Leftrightarrow p\leftrightarrow \neg p\\
                &\Leftrightarrow 0\\
            \end{aligned}
        \]
 

        \item $ ( p \rightarrow q) \wedge  (q \rightarrow r) \wedge  \neg ( p \rightarrow r)$
        \[
            \begin{aligned}
                &\mathrel{\phantom{=}}  ( p \rightarrow q) \wedge  (q \rightarrow r) \wedge  \neg ( p \rightarrow r)\\
                &\Leftrightarrow (\neg p \vee q)\wedge (\neg q \vee r)\wedge \neg (\neg p \vee r)\\
                &\Leftrightarrow (\neg p \vee q)\wedge (\neg q \vee r )\wedge p \wedge \neg r\\
                &\Leftrightarrow ((\neg p \vee q)\wedge p) \wedge ((\neg q \vee r)\wedge \neg r)\\
                &\Leftrightarrow p \wedge q\wedge \neg q \neg r\\
                &\Leftrightarrow 0\\
            \end{aligned}
        \]
    \end{enumerate}
    \item  找出与下列公式等值的尽可能简单的由$\{\neg, \wedge\}$生成的公式:
    \begin{enumerate}
        \item $p \vee q \vee\neg r$
        \[
            \begin{aligned}
                &\mathrel{\phantom{=}} p \vee q \vee\neg r\\
                &\Leftrightarrow \neg(\neg p \wedge \neg q \wedge r)\\
            \end{aligned}        
        \]
        \item $p \vee (\neg  q \wedge  r \rightarrow  p )$
        \[
            \begin{aligned}
                &\mathrel{\phantom{=}} p \vee (\neg  q \wedge  r \rightarrow  p )\\
                &\Leftrightarrow p \vee \neg(\neg q \wedge r)\vee p\\
                &\Leftrightarrow \neg(\neg p \wedge (\neg q \wedge r) \wedge \neg p)\\
                &\Leftrightarrow \neg(\neg p \wedge \neg q \wedge \neg r)\\
            \end{aligned}  
        \]
        \item $p \rightarrow ( q \rightarrow  p)$
        \[
            \begin{aligned}
                &\mathrel{\phantom{=}} p\rightarrow(q\rightarrow p)\\
                &\Leftrightarrow \neg p \vee (\neg q \vee p)\\
                &\Leftrightarrow \neg (p \wedge q \wedge \neg p)\\
                &\Leftrightarrow 1\\
            \end{aligned}    
        \]
    \end{enumerate}
    \item 找出与下列公式等值的尽可能简单的由$\{\neg , \vee\}$生成的公式。
    \begin{enumerate}
        \item  $\neg p \wedge  \neg q \wedge  (\neg r \rightarrow  p)$
        \[
            \begin{aligned}
                &\mathrel{\phantom{=}}\neg p \wedge  \neg q \wedge  (\neg r \rightarrow  p)\\
                &\Leftrightarrow \neg p \wedge \neg q \wedge (r \vee p)\\
                &\Leftrightarrow (\neg p \wedge \neg q \wedge r) \vee (\neg p \wedge q \wedge p)\\
                &\Leftrightarrow \neg p \wedge \neg q \wedge r\\
                &\Leftrightarrow \neg (p \vee q \vee \neg r)\\
            \end{aligned}
        \]
        \item  $( p \rightarrow  q \vee  \neg r) \wedge  \neg p \wedge  q$
        \[
            \begin{aligned}
                &\mathrel{\phantom{=}} ( p \rightarrow  q \vee  \neg r) \wedge  \neg p \wedge  q\\
                &\Leftrightarrow (\neg p \vee q \vee \neg r)\wedge \neg p \wedge \neg q\\
                &\Leftrightarrow \neg(\neg(\neg p \vee q \vee \neg r)\vee p \vee q)\\
            \end{aligned}  
        \]
        \item  $p \wedge  q \wedge  \neg p$
        \[
            \begin{aligned}
                &\mathrel{\phantom{=}} p \wedge  q \wedge  \neg p\\
                &\Leftrightarrow \neg(\neg p \vee \neg q \vee p)\\
                &\Leftrightarrow 0\\
            \end{aligned}
        \]
        
    \end{enumerate}
    \item 设 $A$ 是由$\{\leftrightarrow \}$ 生成的公式。证明:$A$ 是永真式当且仅当每个命题变元在 $A$ 中出现偶数次。
    
    证明:首先证明若$A$中只有一个命题变元$p$时,有:\\
    \[
        A_n\Leftrightarrow\begin{cases}
            1,&\text{$n$是偶数}\\
            p,&\text{$n$是奇数}\\
        \end{cases}
    \]
    记$A_{(p,i)}$表示$p$出现$i$次的公式$A$.\\
    $A_{(p,1)}$中$p$出现了1次,$A_{(p,1)}\Leftrightarrow p$.\\
    $A_{(p,2)}$中$p$出现了2次,$A_{(p,2)}\Leftrightarrow p\leftrightarrow p\Leftrightarrow 1$,是永真式.\\
    设对偶数$n$,题设命题成立,则$A_{(p,n)}\Leftrightarrow 1$,$A_{(p,n+1)}\Leftrightarrow 1\leftrightarrow p\Leftrightarrow p$,$n+1$是奇数,$A_{(p,n+1)}\Leftrightarrow p$.\\
    设对奇数$n$,题设命题成立,则$A_{(p,n)}\Leftrightarrow p$,$A_{(p,n+1)}\Leftrightarrow p\leftrightarrow p\Leftrightarrow 1$,$n+1$是偶数,$A_{(p,n+1)}\Leftrightarrow 1$.\\
    由于$\leftrightarrow$有交换律和分配律成立,对于存在$p_1,p_2,\dots,p_n$n个命题变元,每个命题变元
    分别出现了$k_1,k_2,\dots,k_n$次的公式$A$来说,$A\Leftrightarrow B_{(p_1,k_1)}\leftrightarrow B_{(p_2,k_2)} \leftrightarrow \dots B_{(p_n,k_n)}$。\\
    若$\{k_i\}$都是偶数,则$A\Leftrightarrow B_{(p_1,k_1)}\leftrightarrow B_{(p_2,k_2)} \leftrightarrow \dots B_{(p_n,k_n)}\Leftrightarrow 1\leftrightarrow 1\leftrightarrow \dots 1\Leftrightarrow 1$\\
    若存在$\{k_i\}$的一个子列$\{k_{l_1},k_{l_2},\dots k_{l_m}\}$是奇数,则$A\Leftrightarrow p_{l_1}\leftrightarrow p_{l_2}\leftrightarrow \dots p_{l_m}$不是永真式。\\
    证毕。

    \item 设 $A$ 是由$\{\oplus \}$ 生成的公式。证明:$A$ 是永假式当且仅当每个命题变元在 $A$ 中出现偶数次。
    
    证明:首先证明若$A$中只有一个命题变元$p$时,有:\\
    \[
        A_n\Leftrightarrow\begin{cases}
            0,&\text{$n$是偶数}\\
            p,&\text{$n$是奇数}\\
        \end{cases}
    \]
    记$A_{(p,i)}$表示$p$出现$i$次的公式$A$.\\
    $A_{(p,1)}$中$p$出现了1次,$A_{(p,1)}\Leftrightarrow p$.\\
    $A_{(p,2)}$中$p$出现了2次,$A_{(p,2)}\Leftrightarrow p\oplus p\Leftrightarrow 1$,是永真式.\\
    设对偶数$n$,题设命题成立,则$A_{(p,n)}\Leftrightarrow 0$,$A_{(p,n+1)}\Leftrightarrow 0\oplus p\Leftrightarrow p$,$n+1$是奇数,$A_{(p,n+1)}\Leftrightarrow p$.\\
    设对奇数$n$,题设命题成立,则$A_{(p,n)}\Leftrightarrow p$,$A_{(p,n+1)}\Leftrightarrow p\oplus p\Leftrightarrow 0$,$n+1$是偶数,$A_{(p,n+1)}\Leftrightarrow 0$.\\
    由于$\oplus$有交换律和分配律成立,对于存在$p_1,p_2,\dots,p_n$n个命题变元,每个命题变元
    分别出现了$k_1,k_2,\dots,k_n$次的公式$A$来说,$A\Leftrightarrow B_{(p_1,k_1)}\oplus B_{(p_2,k_2)} \oplus \dots B_{(p_n,k_n)}$。\\
    若$\{k_i\}$都是偶数,则$A\Leftrightarrow B_{(p_1,k_1)}\oplus B_{(p_2,k_2)} \oplus \dots B_{(p_n,k_n)}\Leftrightarrow 0\oplus 0\oplus \dots 0\Leftrightarrow 0$\\
    若存在$\{k_i\}$的一个子列$\{k_{l_1},k_{l_2},\dots k_{l_m}\}$是奇数,则$A\Leftrightarrow p_{l_1}\oplus p_{l_2}\oplus \dots p_{l_m}$不是永真式。\\
    证毕。
    \item 北京、上海、天津、广州四市乒乓球队比赛,三个观众猜测比赛结果:

    甲说:“天津第一,上海第二。”
    
    乙说:“天津第二,广州第三。”
    
    丙说:“北京第二,广州第四。”
    
    比赛结果显示,每人猜对了一半,并且没有并列名次。问:实际名次怎样排列?
    
    答:天津第一,北京第二,广州第三,上海第四。

    \item 某勘探队取回一块矿样,三人判断如下。

    甲说:“矿样不含铁,也不含铜。”
    
    乙说:“矿样不含铁,含锡。”
    
    丙说:“矿样不含锡,含铁。”
    
    已经知道,这三人中有一个是专家,一个是老队员,一个是实习队员。化验结果表明:这块矿样只含一种金属,专家的两个判断皆对,老队员的判断一对一错,实习队员的两个判断皆错。问:这三人的身分各是什么?
    
    答:丙是专家,甲是老队员,乙是实习队员。
    
    \item 【选做】先用等值演算证明下列等值式,再用对偶定理得出新等值式。
    \begin{enumerate}
        \item  $\neg (\neg p \vee  \neg q) \vee  \neg (\neg p \vee  q) \Leftrightarrow  p$
        \[
            \begin{aligned}
                &\mathrel{\phantom{=}} \neg (\neg p \vee  \neg q) \vee  \neg (\neg p \vee  q) \Leftrightarrow  p\\
                &\Leftrightarrow (p \wedge q)\vee(p\wedge \neg q)\\
                &\Leftrightarrow p\wedge (q\vee \neg q)\\
                &\Leftrightarrow p\wedge 1\\
                &\Leftrightarrow p\\
            \end{aligned}
        \]
        用对偶定理得出的新等值式为:
        \[
            \neg(\neg p\wedge \neg q)\wedge \neg (\neg p \wedge q)\Leftrightarrow p
        \]
    
        \item  $( p \vee  \neg q) \wedge  ( p \vee  q) \wedge  (\neg p \vee  \neg q) \Leftrightarrow  \neg (\neg p \vee  q)$
        \[
            \begin{aligned}
                &\mathrel{\phantom{=}} ( p \vee  \neg q) \wedge  ( p \vee  q) \wedge  (\neg p \vee  \neg q) \Leftrightarrow  \neg (\neg p \vee  q)\\
                &\Leftrightarrow p\vee (\neg q \wedge q) \wedge(\neg p \vee \neg q)\\
                &\Leftrightarrow p\wedge (\neg q \wedge \neg q)\\
                &\Leftrightarrow (p \wedge \neg p)\vee (p \wedge \neg q)\\
                &\Leftrightarrow p \wedge \neg q\\
                &\Leftrightarrow \neg(\neg p \vee q)\\
            \end{aligned}
        \]
        用对偶定理得出的新等值式为:
        \[
            ( p \wedge  \neg q) \vee  ( p \wedge  q) \vee  (\neg p \wedge  \neg q) \Leftrightarrow  \neg (\neg p \wedge  q)
        \]
        
        \item $ q \vee  \neg ((\neg p \vee  q) \wedge  p) \Leftrightarrow  1$
        \[
            \begin{aligned}
                &\mathrel{\phantom{=}} q \vee  \neg ((\neg p \vee  q) \wedge  p)\\
                &\Leftrightarrow q\vee \neg(\neg p\vee q) \vee \neg p\\
                &\Leftrightarrow q\vee (p \wedge \neg q) \vee \neg p\\
                &\Leftrightarrow (q\vee p)\wedge(q \vee \neg q)\vee \neg p\\
                &\Leftrightarrow q \vee p \vee \neg p\\
                &\Leftrightarrow 1
            \end{aligned}
        \]
        用对偶定理得出的新等值式为:
        \[
            q \wedge  \neg ((\neg p \wedge  q) \vee  p) \Leftrightarrow  1
        \]
    \end{enumerate}
    \item 【选做】设 A 是由$\{0,1,\neg, \vee, \wedge \}$生成的公式, A* 与 A 互为对偶式。证明:
    \begin{enumerate}
        \item 若 $A$ 是永真式,则 $A*$ 是永假式。
        
        证:由$A$永真,有$A\Leftrightarrow 1$,应用对偶定理,有$A*\Leftrightarrow 0$,证毕。
     
        \item 若 A 是永假式,则 A* 是永真式。

        证:由$A$永假,有$A\Leftrightarrow 0$,应用对偶定理,有$A*\Leftrightarrow 1$,证毕。
    \end{enumerate}


     

    \end{enumerate}
\end{document}