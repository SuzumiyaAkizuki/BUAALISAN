\documentclass[UTF8]{ctexart}
\usepackage{float}
\usepackage{listings}
\usepackage{graphicx}
\usepackage{xeCJK}
\usepackage{xcolor}
\usepackage{ulem}
\usepackage{CJKfntef}
\usepackage{setspace}
\usepackage{amsmath}
\lstset{
   basicstyle          =   \ttfamily,          % 基本代码风格
   keywordstyle        =   \ttfamily\bfseries,          % 关键字风格
   commentstyle        =   \rmfamily\itshape,  % 注释的风格,斜体
   stringstyle         =   \ttfamily,  % 字符串风格
   flexiblecolumns,                % 别问为什么,加上这个
   numbers             =   left,   % 行号的位置在左边
   showspaces          =   false,  % 是否显示空格,显示了有点乱,所以不现实了
   numberstyle         =   \zihao{-5}	\ttfamily ,    % 行号的样式,小五号,tt等宽字体
   showstringspaces    =   false,
   captionpos          =   t,      % 这段代码的名字所呈现的位置,t指的是top上面
   frame               =   lrtb,   % 显示边框
   breaklines      =   true,   % 自动换行,建议不要写太长的行
   columns         =   fixed,  % 如果不加这一句,字间距就不固定,很丑,必须加
}

\title{离散数学第九次作业}
\author{梁秋月}
\date{2021年5月29日}

\begin{document}
    \maketitle
    \begin{enumerate}
        \item 判断以下逻辑推论关系是否成立,并说明理由。
        \begin{enumerate}
            \item $\forall x(P(x) \vee  Q(x)) \models  \forall xP(x) \vee  \forall xQ(x)$
            
            不成立,取解释$I_1$:
            \[
                D_\{I_1\}={0,1},P(x)=x,Q(x)=\neg x
            \]
            则$I(\forall x(P(x)\vee Q(x)))(v)=1,I(\forall xP(x)\vee \forall xQ(x))(v)=0$,推论关系不成立。

            \item $\exists xP(x) \wedge  \exists xQ(x) \models  \exists x(P(x) \wedge  Q(x))$
            
            不成立。对上述$I_1$,$I(\exists xP(x)\wedge \exists xQ(x))(v)=1$,$\exists x(P(x)\wedge Q(x))(v)=0$
            ,推论关系不成立。

            \item $\forall x(P(x) \leftrightarrow  \forall xQ(x)) \models  \forall x(P(x) \leftrightarrow  Q(x))$
            
            不成立。取解释$I_3$:
            \[
                D_{I_3}=\{0,1\},P(x)=0,Q(x)=x
            \]
            则$I(\forall x(P(x)\leftrightarrow \forall xQ(x)))(v)=1$,$I(\forall x(P(x)\leftrightarrow Q(x)))(v)=0$
            ,推论关系不成立。
            
            \item $\forall x(P(x) \rightarrow  \forall xQ(x)) \models  \forall x(P(x) \rightarrow  Q(x))$
            
            成立。假如解释$I_4$使得$I(\forall x(P(x)\rightarrow Q(x)))(v)=0$,那么$\exists d\in D_{I_4},I(P(x)\rightarrow Q(x))(v[x/d])=0$,
            即$I(P(x))(v[x/d])=1 \text{ and }I(Q(x))(v[x/d])=0$,则$I(P(x)\rightarrow \forall xQ(x))(v[x/d])=0$,推论前件为0。

            
            \item $\exists x(P(x) \rightarrow  Q(x)), \exists xP(x) \models  \exists xQ(x)$
            
            不成立。取解释$I_5$:
            \[
                D_{I_5}=\{0,1\},P(x)=x,Q(x)=0
            \]
            则$I(\exists x(P(x)\rightarrow Q(x)))(v)=1,I(\exists xP(x))(v)=1 \text{ and } I(\exists xQ(x))(v)=0$,
            推论关系不成立。

            \item $\exists x\exists yP(x, y) \models  \exists xP(x, x)$
            
            不存在。取解释$I_6$:
            \[
                D_{I_6}=R,P(x,y)=(x=y)
            \]
            则$I(\exists x\exists yP(x,y))(v)=1,I(\exists xP(x,x))(v)=0$,逻辑推论不成立。
            
        \end{enumerate}
        
        \item 设 A,B 是任意公式,证明以下结论。
        \begin{enumerate}
            \item $\exists x( A \wedge  B) \models  \exists xA \wedge  \exists xB$
            
            假设存在赋值$v$和解释$I$使得$I(\exists x(A\wedge B))(v)=1$,则:
            \[
                \begin{aligned}
                    &I(\exists x(A\wedge B))(v)=1\\
                    &\Rightarrow \exists d\in D_I,I(A\wedge B)(v[x/d])=1\\
                    &\Rightarrow \exists d\in D_I,I(A)(v[x/d])=I(B)(v[x/d])=1\\
                    &\Rightarrow I(\exists x A)(v)=I(\exists xB)(v)=1\\
                    &\Rightarrow I(\exists x A\wedge \exists xB)(v)=1\\
                \end{aligned}
            \]
            故推论关系成立。

            \item $\forall x( A \rightarrow  B), \forall xA \models  \forall xB$
            
            假设存在解释$I$和赋值$v$使得$I(\forall xA)(v)=1,I(\forall xB)(v)=0$.
            则$\exists d\in D_I,I(B)(v[x/d])=0 \text{ and } I(A)(v[x/d])=1$,则$I(A\rightarrow B)(v[x/d])=0$,
            则$I(\forall x(A\rightarrow B))(v)=0$,故推论关系成立。
            
            \item  $\exists x A_x^y \models  \exists x\exists y A$ ,其中 x 对于 A 中的 y 是可代入的。
            
            假设解释I和赋值v使得$I(\exists x A_x^y)(v)$成立,那么有$\exists d\in D_I,I(A_x^y)(v[x/d])=1$
            由于$I(A_x^y)(v[x/d])=I(A[x/d,y/d])=1$,所以推论成立。
            
            \item  $\exists x A \rightarrow  \exists xB \models  \exists x( A \rightarrow  B)$
        
            假设解释I和赋值v使得$I(\exists x(A\rightarrow B)$,有
            \[
                \begin{aligned}
                    &I(\exists x(A\rightarrow B))(v)=0\\
                    &\Rightarrow \forall d\in D_I, I(B)(v[x/d])=0 \text{ and }I(A)(v[x/d])=1\\
                    &\Rightarrow I(\exists xB)(v)=0\text{ and } I(\exists xA)(v)=1
                \end{aligned}
            \]
            故推论关系成立。
        
        \end{enumerate}
        
        \item 设变元 x 既不是公式 B 中的自由变元,也不是公式集 $\Gamma$ 中任何公式的自由变元,A 是公式。若$ \Gamma \cup {A} \models  B $,则$\Gamma  \cup {\exists x A} \models  B$ 。

        假设解释$I$和赋值$v$满足$\Gamma\cup \{\exists xA\}$,则
        $\exists d\in D_I,I(A)(v[x/d])=1$,显然I和v[x/d]也满足$\Gamma$,因为其中x不是自由变元。

        那么I和v[x/d]满足$\Gamma \cup A$,又因为$\Gamma \cup {A} \models  B$,所以$I(B)(v[x/d])=1$,
        又因为x不是B的自由变元,所以$I(B)(v)=1$,所以$\Gamma \cup \{\exists A\}\models B$

        \item 设$\Gamma  $是公式集合,A 是公式,则$\Gamma  \models  A$ 当且仅$当\Gamma  \cup {\neg  A}$ 不可满足。

        \begin{enumerate}
            \item 假设解释I,赋值v使得$\Gamma \cup \{\neg A\}$满足,那么$I(A)(v)=0,\Gamma$不能推出A。
            \item 假设$\Gamma $不能推导出A,那么存在解释和赋值使得$\Gamma $满足且$I(A)(v)=0$,那么$\Gamma \cup \{\neg A\}$满足
        \end{enumerate}

        \item 判断以下公式集合是否可满足,并说明理由。
        \begin{enumerate}
            \item $\{\neg P(t) | \text{ t是项 }\}\cup {\exists xP(x)}$
            
            可满足。取解释:
            \[
                D_I=\{0,1\},P(x)=x
            \]
            赋值:
            对每个常元a有$a=1$\\
            对每个变元x有$x=1$\\
            对每个谓词f,$f(x_1,\cdots ,x_n)=1$\\
            
            \item ${\forall x \neg P(x, x), \forall x\forall y\forall z(P(x, y) \wedge  P( y, z) \rightarrow  P(x, z)), \forall x\exists yP(x, y)}$
            可满足。取解释:
            \[
                D_I=R,P(x,y)=(x<y)
            \]
        \end{enumerate}

        
        
        
        
        
    \end{enumerate}
\end{document}
