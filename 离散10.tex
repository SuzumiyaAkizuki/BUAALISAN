\documentclass[UTF8]{ctexart}
\usepackage{float}
\usepackage{listings}
\usepackage{graphicx}
\usepackage{xeCJK}
\usepackage{xcolor}
\usepackage{ulem}
\usepackage{CJKfntef}
\usepackage{setspace}
\usepackage{amsmath}
\usepackage{amssymb}
\usepackage{mathrsfs}
\lstset{
   basicstyle          =   \ttfamily,          % 基本代码风格
   keywordstyle        =   \ttfamily\bfseries,          % 关键字风格
   commentstyle        =   \rmfamily\itshape,  % 注释的风格,斜体
   stringstyle         =   \ttfamily,  % 字符串风格
   flexiblecolumns,                % 别问为什么,加上这个
   numbers             =   left,   % 行号的位置在左边
   showspaces          =   false,  % 是否显示空格,显示了有点乱,所以不现实了
   numberstyle         =   \zihao{-5} \ttfamily ,    % 行号的样式,小五号,tt等宽字体
   showstringspaces    =   false,
   captionpos          =   t,      % 这段代码的名字所呈现的位置,t指的是top上面
   frame               =   lrtb,   % 显示边框
   breaklines      =   true,   % 自动换行,建议不要写太长的行
   columns         =   fixed,  % 如果不加这一句,字间距就不固定,很丑,必须加
}

\title{离散数学第十次作业}
\author{贠启豪\\19375168}
\date{\today}
\newtheorem{thm}{定理}[section]
\newtheorem{define}{定义}[section]
\def\RR{{\mathbb{R}}}
\def\AA{{\mathscr{A}}}
\def\And{\wedge}
\def\To{\rightarrow}
\def\Or{\vee}
\def\Not{\neg}

\begin{document}
    \maketitle

    求证:
    \begin{enumerate}
        \item $\vdash (P\To(Q\To R))\To (Q\To (P\To R))$
        \begin{flalign*}
            A_1 &= (P\To (Q\To R))\To ((P\To Q)\To(P\To R)) & \AA_2\\
            A_2 &= ((P\To Q)\To(P\To R)) \To (Q\To ((P\To Q)\To(P\To R))) & \AA_1\\
            A_3 &= (Q\To ((P\To Q)\To(P\To R))) \To ((Q\To (P\To Q))\To (Q\To (P\To R))) &\AA_2\\
            A_4 &= (P\To (Q\To R))\To((Q\To (P\To Q))\To (Q\To (P\To R))) & MP\\
            A_5 &= (P\To (Q\To R))\To((Q\To (P\To Q))\To (Q\To (P\To R))) &\quad\\
                &\To ((P\To (Q\To R))\To(Q\To (P\To Q)))\To((P\To (Q\To R))\To(Q\To (P\To R))) &\AA_2\\
            A_6 &= ((P\To (Q\To R))\To(Q\To (P\To Q)))\To((P\To (Q\To R))\To(Q\To (P\To R))) & MP\\
            A_7 &= Q\To (P\To Q) & \AA_1\\
            A_8 &= (Q\To (P\To Q))\To ((P\To (Q\To R)) \To (Q\To (P\To Q))) & \AA_1\\
            A_9 &= (P\To (Q\To R)) \To (Q\To (P\To Q)) & MP\\
            A_{10} &= (P\To(Q\To R))\To (Q\To (P\To R)) & MP\\
        \end{flalign*}

        \item $\vdash (Q\To R)\To ((P\To Q)\To (P\To R))$
        \begin{flalign*}
            A_1 &= (Q\To R)\To (P\To (Q\To R)) & \AA_1\\
            A_2 &= (P\To (Q\To R))\To ((P\To Q)\To (P\To R)) & \AA_2\\
            A_3 &= (Q\To R)\To ((P\To Q)\To (P\To R)) & MP\\
        \end{flalign*}

        \item $\vdash (P\To Q)\To ((Q\To R)\To (P\To R))$
        \begin{flalign*}
            A_1 &= (Q\To R)\To ((P\To Q)\To (P\To R)) & Q_2\\
            A_2 &= ((Q\To R)\To ((P\To Q)\To (P\To R))) &\quad \\
                &\To ((P\To Q)\To ((Q\To R)\To (P\To R))) & Q_1\\
            A_3 &= (P\To Q)\To ((Q\To R)\To (P\To R)) & MP\\
        \end{flalign*}

        \item $\vdash Q\To Q$
        \begin{flalign*}
            A_1 &= Q\To ((Q\To Q))\To Q) &\AA_1\\
            A_2 &= (Q\To ((Q\To Q))\To Q))\To ((Q\To (Q\To Q))\To (Q\To Q)) & \AA_2\\
            A_3 &= (Q\To (Q\To Q))\To (Q\To Q) & MP\\
            A_4 &= Q\To (Q\To Q) & \AA_1\\
            A_5 &= Q\To Q & MP\\
        \end{flalign*}

        \item $\vdash \neg \neg Q\To Q$
        \begin{flalign*}
            A_1 &= \Not \Not Q\To(\Not \Not \Not \Not Q \To \Not \Not Q) & \AA_1\\
            A_2 &= (\Not \Not \Not \Not Q\To \Not \Not Q)\To (\Not Q\To \Not \Not \Not Q) & \AA_3\\
            A_3 &= (\Not Q\To \Not \Not \Not Q)\To (\Not \Not Q \To Q) & \AA_3\\
            A_4 &= (\Not \Not Q)\To (\Not \Not Q\To Q) & MP\\
            A_5 &= ((\Not \Not Q)\To (\Not \Not Q\To Q))\\
                &\To ((\Not \Not Q\To \Not \Not Q)\To (\Not \Not Q\To Q)) & \AA_2\\
            A_6 &= (\Not \Not Q\To \Not \Not Q)\To (\Not \Not Q\To Q) & MP\\
            A_7 &= \Not \Not Q\To \Not \Not Q & Q_4\\
            A_8 &= \Not \Not Q\To Q & MP\\
        \end{flalign*}

        \item $\vdash Q\To \Not \Not Q$
        \begin{flalign*}
            A_1 &= (\Not \Not \Not Q \To \Not Q)\To (Q\To \Not \Not Q) & \AA_3 \\
            A_2 &= (\Not \Not \Not Q \To \Not Q) & Q_5\\
            A_3 &= Q\To \Not \Not Q & MP\\
        \end{flalign*}

        \item $\vdash (\Not \Not Q\To \Not \Not R)\To (Q\To R)$
        \begin{flalign*}
            A_1 &= (\Not \Not Q \To \Not \Not R)\To (\Not R \To \Not Q) & \AA_3\\
            A_2 &= (\Not R \To \Not Q)\To (Q \To R) & \AA_3\\
            A_3 &= (\Not \Not Q\To \Not \Not R)\To (Q\To R) & MP\\
        \end{flalign*}

        \item $\vdash (\Not Q\To Q)\To (R\To Q)$
        \begin{flalign*}
            A_1 &= \Not Q \To (\Not \Not R \To \Not Q) & \AA_1\\
            A_2 &= (\Not \Not R \To \Not Q)\To (Q\To \Not R) & \AA_3\\
            A_3 &= \Not Q \To (Q\To \Not R) & MP\\
            A_4 &= (\Not Q \To (Q\To \Not R))\To ((\Not Q \To Q)\To (\Not Q \To \Not R)) & MP\\
            A_5 &= (\Not Q \To Q)\To (\Not Q \To \Not R) & MP\\
            A_6 &= (\Not Q\To \Not R)\To (R \To Q) & \AA_3\\
            A_7 &= (\Not Q \To Q)\To (R\To Q) & MP\\
        \end{flalign*}

        \item $\vdash Q \To (R \Or Q)$
        \begin{flalign*}
            A_1 &= Q\To (\Not R \To Q) & \AA_1 \\
            A_2 &= (\Not R \To Q)\Leftrightarrow (R\vee Q) & \text{定义}\\
            A_3 &= Q \To (R \Or Q)\\
        \end{flalign*}

        \item $(Q\And R)\To Q$
        \begin{flalign*}
            A_1 &= \Not Q\To (R\To \Not Q) & \AA_1\\
            A_2 &= (R \To \Not Q)\To (Q \To \Not R) & \AA_3 \\
            A_3 &= \Not Q \To (Q\To \Not R) & MP\\
            A_4 &= (\Not Q \To (Q\To \Not R))\To (\Not (Q\To \Not R)\To Q) & \AA_3 \\
            A_5 &= \Not (Q\To \Not R)\To Q & MP\\
            A_6 &= (Q \And R)\To Q & \text{定义}\\
        \end{flalign*}

        \item $\vdash Q\To (\Not R\To \Not (Q\To R))$
        \begin{flalign*}
            A_1 &= (Q\To R)\To (Q\To R) & Q_4 \\
            A_2 &= ((Q\To R)\To (Q\To R))\To (Q \To ((Q\To R)\To R) & Q_1\\
            A_3 &= Q \To ((Q\To R)\To R) & MP\\
            A_4 &= ((Q\To R)\To R) \To (\Not R \To \Not (Q\To R)) & \AA_3\\
            A_5 &= Q \To (\Not R\To \Not (Q\To R)) & MP\\
        \end{flalign*}

        \item $\vdash (P\To R)\And (Q\To S)\To (P\And Q\To R\And S)$

        由演绎定理,此公式可以写成:$(P\To R)\And (Q\To S),(P\And Q)\vdash R\And S\\$

        \begin{flalign*}
            A_1 &= (P\To R)\And (Q\To S) & \Gamma \\
            A_2 &= (P\To R)\And (Q\To S) \To (P\To R) & Q_{10}\\
            A_3 &= P\To R & MP\\
            A_4 &= (P\To R)\And (Q\To S) \To (Q\To S) & Q_{10}\\
            A_5 &= Q\To S & MP\\
            A_6 &= P\And Q & \Gamma\\
            A_7 &= (P\And Q)\To P & Q_{10}\\
            A_8 &= P & MP\\
            A_9 &= (P\And Q)\To Q & Q_{10}\\
            A_{10} &= Q & MP\\
            A_{11} &= R & MP\\
            A_{12} &= S & MP\\
            A_{13} &= R\And S 
        \end{flalign*}
    \end{enumerate}
    
\end{document}